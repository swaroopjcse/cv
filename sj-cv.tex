\documentclass[11pt,a4paper]{moderncv}

%% ModernCV themes
% \moderncvstyle{banking}
% \moderncvstyle{casual}
\moderncvstyle{classic}
\moderncvcolor{red}
\renewcommand{\familydefault}{\sfdefault}
% \nopagenumbers{}

%%------------------------------------------------------------------------------
\usepackage[sorting=ydnt, backend=biber, style=numeric, giveninits=true,
maxbibnames=99, defernumbers=true]{biblatex}
\DefineBibliographyStrings{english}{%
  mathesis = {Master's Project Report},%
}

\defbibenvironment{bibliography}
  {\list
     {\printtext[labelnumberwidth]{% label format from numeric.bbx
        \printfield{prefixnumber}%
        \printfield{labelnumber}}}
     {\setlength{\topsep}{0pt}% layout parameters from moderncvstyleclassic.sty
      \setlength{\labelwidth}{\hintscolumnwidth}%
      \setlength{\labelsep}{\separatorcolumnwidth}%
      \leftmargin\labelwidth%
      \advance\leftmargin\labelsep}%
      \sloppy\clubpenalty4000\widowpenalty4000}
  {\endlist}
  {\item}

\addbibresource{/Users/swaroop/Documents/Professional/mypubs/mypub.bib}

%%------------------------------------------------------------------------------

\usepackage{academicons}
\usepackage{fontawesome}
\usepackage[none]{hyphenat} % avoid hyphens
%%------------------------------------------------------------------------------
\renewcommand*{\mobilephonesymbol}    {{\faPhone}~}
\renewcommand*{\emailsymbol}          {{\faicon{envelope-o}~}}
\renewcommand*{\homepagesymbol}       {{\faicon{globe}~}}
\renewcommand*{\linkedinsocialsymbol} {{\faicon{linkedin}~}}
\renewcommand*{\addresssymbol}        {{\faicon{university}~}}
\renewcommand*{\twittersocialsymbol}  {{\faicon{twitter}~}}
\renewcommand*{\githubsocialsymbol}   {{\faicon{github}~}}
%%------------------------------------------------------------------------------
\definecolor{color1}{RGB}{187,0,0}% OSU Scarlet
\definecolor{color2}{RGB}{102,102,102}% OSU Gray
%%------------------------------------------------------------------------------

%% Character encoding
\usepackage[utf8]{inputenc}

%% Adjust the page margins
\usepackage[scale=0.75]{geometry}
\usepackage{fontspec}

%% Personal data
\firstname{Swaroop}
\familyname{Joshi}
% \title{Senior Lecturer\\\cse\\\osu}
% \address{395 Dreese Labs, 2015 Neil Ave}{Columbus OH 43210}
\address{405 Caldwell Labs\\2024 Neil Ave\\}{Columbus OH 43210}
% \phone{+2~(345)~678~901}
% \fax{+3~(456)~789~012}
\email{joshi.127@osu.edu}
\homepage{go.osu.edu/swaroop}
% \mobile{+1~(917)~833~8905}
% \extrainfo{Updated: \VCDateRAW}
% \photo[64pt][0.4pt]{picture}
% \quote{Researcher, Developer, Educator}
\social[linkedin]{swaroop84}
% \social[github]{swaroopjcse}


%%------------------------------------------------------------------------------

\usepackage{fancyhdr}
\usepackage{lastpage}
\fancyfoot[L]{\VCAuthor}
\fancyfoot[R]{\footnotesize Page \thepage\ of \pageref{LastPage}}
\fancypagestyle{firststyle}
{
   \fancyhf{}
   \fancyfoot[L]{\footnotesize Updated: \VCDateRAW~(\VCRevision)}
   \fancyfoot[R]{\footnotesize Page \thepage\ of \pageref{LastPage}}
}
%%------------------------------------------------------------------------------

%% My macros
\newcommand{\con}{CONSIDER}
\newcommand{\confull}{CONflicting Student Ideas Discussed, Evaluated and Resolved}
\newcommand{\osu}{The Ohio State University}
\newcommand{\iitb}{Indian Institute of Technology Bombay}
\newcommand{\iitbs}{IIT Bombay}
\newcommand{\grc}{GCC Resource Center}
\newcommand{\nitk}{National Institute of Technology Karnataka}
\newcommand{\nitks}{NITK}
\newcommand{\cse}{Computer Science \& Engineering}
\newcommand{\gra}{Graduate Research Associate}
\newcommand{\gta}{Graduate Teaching Associate}
\newcommand{\geogame}{\url{geogame.osu.edu}}
\newcommand{\thesis}{\con: A Novel Approach to Conflict-Driven Collaborative-Learning}
\newcommand{\gdfa}{Extending the Generic Data-Flow Analyzer (gdfa) in GCC}
\newcommand{\ucat}{University Center for Advancement in Teaching}
\newcommand{\asee}{American Society for Engineering Education}
\newcommand{\iucee}{Indo-Universal Collaboration for Engineering Education}
%%------------------------------------------------------------------------------

\immediate\write18{sh ./vc}
\input{vc}
%%------------------------------------------------------------------------------
%% Content
%%------------------------------------------------------------------------------
\begin{document}
\makecvtitle
\thispagestyle{firststyle}
\section{Experience}
% \label{sec:Experience}
\cventry{}{Senior Lecturer}{}{\cse, \osu}{}{January, 2017 -- present}
\cventry{}{Graduate Teaching and Research Associate}{}{\cse, \osu}{}{2011--16}
\cventry{}{Senior Project Engineer}{}{\grc, \iitb}{}{2010--11}
\cventry{}{Teaching Assistant}{}{\cse, \iitb}{}{2008--10}
\cventry{}{Software Engineer}{}{SoftJin Technologies Pvt.\ Ltd., Bangalore India}{}{2005--06}


\section{Education}
\cventry{2017}{Doctor of Philosophy}{\osu}{Columbus OH}{}{\cse}  % arguments 3 to 6 can be left empty
\cvitem{thesis}{\emph{\thesis}}
\cventry{2016}{Master of Science}{\osu}{Columbus OH}{}{\cse}
\cventry{2010}{Master of Technology}{\iitb}{Mumbai India}{}{\cse}
\cvitem{project}{\emph{\gdfa}}
\cventry{2005}{Bachelor of Engineering}{\nitk}{India}{}{Computer Engineering}


\section{Awards and Honors}
\cventry{}{Lecturer Teaching Development Grant}{Spring 2017}{}{}{\ucat, \osu}%
\cventry{}{Best Student Paper Award}{2016}{}{}{\asee, North Central Section}%
\clearpage


%%------------------------------------------------------------------------------
% \section{Research Experience}
% \subsection{Ph.D.\ Thesis}
% \cvitem{title}{\emph{\thesis}}
% \cvitem{advisor}{Prof.\ Neelam Soundarajan, \cse, OSU}
% \cvitem{\faicon{globe}}{\url {http://rave.ohiolink.edu/etdc/view?acc_num=osu1500511662959839}}
% \cvitem{description}{Piaget's classic work on cognitive development showed that
  % engaging learners in critical discussions with peers about ideas that are
  % different than theirs leads to deep conceptual understanding. I have developed
  % a highly innovative collaborative learning approach that exploits specific
  % affordances of web technologies to incorporate Piaget's well established ideas
  % into college-level STEM education. This approach, named CONSIDER (short for
  % CONflicting Ideas Discussed, Evaluated, and Resolved) allows creation of small
  % groups of students with different ideas about the topic in question, engages
  % them in a highly-structured rounds-based discussion so that the group
  % progresses at an equitable pace, and makes their submissions anonymous to
  % others so that students can receive the comments without any preconceived
  % notions they may have about the poster. A platform independent responsive web
  % application was developed to implement this approach and to compare it with
  % the commonly used discussion forum approach, where students respond to other
  % posts in a threaded discussion. A quantitative gain-score analysis of pre- and
  % post-test scores shows that students developed better understanding of the
  % topics discussed with the CONSIDER approach, than they did with the
  % conventional, forum-based approach.}

% \subsection{Other Research Projects}
% \cventry{2014--2016}{GeoGame}{\osu}{}{}{}

% \cvitem{}{Developed functionality for GeoGame, a web based role-playing game
  % developed using MVC .NET, which lets the students of geography experience the
  % life of an ordinary farmer in developing countries.}

% \cvitem{}{Coordinated the teams of developers, adding functionality to the game,
  % designing experiments to evaluate effects of the game on learning, and
  % resolving software issues.}

% \cvitem{}{Led multiple capstone teams during the course of these projects.}
% \cvitem{\faicon{globe}}{\url{http://geogame.osu.edu}}

% \cventry{2011--2012}{PolyOpt/Fortran}{\osu}{}{}{}%
% \cvitem{}{Worked on testing and evaluation of a software PolyOpt/Fortran, which
  % is a locally developed loop optimization framework}%
% \cvitem{\faicon{globe}}{\url{http://web.cse.ohio-state.edu/~pouchet.2/software/polybench/polybench-fortran.html}}


% \cventry{2010--2011}{Generic Data-Flow Analyzer}{\iitbs}{}{}{}%
% \cvitem{}{Adapted the Generic Data-Flow Analyzer in GCC 4.6, making it available
  % as a plugin while working on my master's project and then as a Senior Project
  % Engineer at the \grc}%
% \cvitem{\faicon{globe}}{\url{http://www.cse.iitb.ac.in/grc/}}

% \clearpage
% \cvitem{\aiicon{orcid}}{\url{http://orcid.org/0000-0003-4536-2446}}
%%------------------------------------------------------------------------------
% \renewcommand{\refname}{Some Relevant Publications}
\section{Publications}
\nocite{*}
\printbibliography[title={Book Chapters}, type=incollection, heading=subbibliography]%
\printbibliography[title=Conference Proceedings, type=inproceedings, heading=subbibliography]%
\printbibliography[title=Dissertations, keyword={thesis}, heading=subbibliography]%
\printbibliography[title=Other Publications, keyword={tr}, heading=subbibliography]%
%%------------------------------------------------------------------------------
\section{Teaching Experience}
\cventry{}{Principles of Programming Languages}{OSU}{}{}{Spring, Fall 2018}%
\cventry{}{Software II: Software Development and Design}{OSU}{}{}{Fall 2018}%
\cventry{}{Software I: Components}{OSU}{}{}{Spring, Summer, Fall 2017; Spring 2018}%
\cventry{}{Introduction to Computer Programming In Java}{OSU}{}{}{Summer 2017}%
\cventry{}{Data Structures Using Java}{OSU}{}{}{Fall 2016}%
\cventry{}{Mobile App Development}{OSU}{}{}{Spring and Summer 2014}%
\cventry{}{Software-I}{OSU}{(Grader)}{}{Fall 2014}%
\cventry{}{C++ Programming}{OSU}{}{}{Summer 2013}%
\cventry{}{Introduction to Programming}{OSU}{}{}{Spring 2013}%
\cventry{}{Operating Systems}{OSU}{(Grader)}{}{Summer 2013}%
\cventry{}{Program Analysis}{\iitbs}{(Grader)}{}{2009--2010}%
\cventry{}{Introduction to programming}{\iitbs}{(Grader)}{}{2008}%
% \clearpage

% ------------------------------------------------------------------------------
\section{Talks/Presentations}
\cventry{Jul.\ 2018}{Cooperative and Collaborative Learning in Engineering Classrooms}{\iucee}{Webinar}{}{}%
\cventry{Aug.\ 2016}{New TA Orientation Facilitator}{\ucat}{\osu}{}{}%

\section{Workshops attended}
\cventry{Mar.\ 2016}{DO-IT Access Cyberlearning}{University of Washington,
  Seattle}{}{}{}

\clearpage
%%------------------------------------------------------------------------------
\section{Service Activities}
\subsection{Field/Profession}
\cventry{2018--present}{Associate Editor}{Journal of Engineering Education
  Transformations}{}{}{}%
\cventry{2018--present}{Journal Reviewer}{}{}{}{}%
\cvitem{}{The ASEE Computers in Education (COED) Journal}%
\cventry{2016--present}{Technical Program Committees}{}{}{}{}%
% \cvitem{}{\href{http://icatse.org/icisa2018/}{9th iCatse Conference on
    % Information Science and Applications (ICISA 2018)}}%
\cvitem{}{ACM SIGCSE Technical Symposium on Computer Science Education, 2019, 2018}%
\cvitem{}{ASEE Annual Conference \& Exposition, 2018, 2017, 2016}%
\cvitem{}{IEEE Frontiers in Education, 2018, 2017}%
% \cvitem{}{\href{http://news.uitm.edu.my/index.php/n/research/247-8th-international-conference-on-university-learning-and-teaching-incult-2016}{8th
    % International Conference on University Learning and Teaching (InCULT
    % 2016)}}%
\cventry{2018--20}{Secretary-Treasurer}{Computers in Education Division}{ASEE}{}{}%
\cventry{2016}{Volunteer}{Buck-i-code}{ACM-W, OSU}{}{A day-long workshop
  targeted at school-going female students to get them interested in computer
  science at a young age.}%
\cventry{2009, 2010}{Technical assistant}{\grc}{\iitbs}{}{The \grc\ organizes a
  workshop ``Essential Abstractions in GCC'' every year in July. I was part of
  the technical team, in charge of conducting the lab sessions for two
  workshops.}%


\subsection{University}
\cventry{2018}{Research Poster Judge}
{\href{https://cgs.osu.edu/hayes-forum/}{32nd Edward F. Hayes Graduate Research
    Forum}}{OSU}{}{}%
\cventry{2017}{Faculty Mentor}{New Graduate TA Orientation}{\ucat}{\osu}{}%
\cventry{2015--2017}{Graduate Representative}{Mobile Governance}{OSU}{}{I was
  part of this collaborative effort between OCIO, TCO, and other stakeholders at
  OSU to ensure the large number of mobile and web applications developed by
  various departments at OSU (e.g.\ medical center) meet the quality and
  branding standards of the university.}%
\cventry{2015}{Research Poster Judge}{20th Denman Undergraduate Research
  Forum}{OSU}{}{}%


%%------------------------------------------------------------------------------

\section{Softwares Developed}

\cventry{\faicon{code}}{\con}{}{}{}{A web app, developed using Google App Engine
  and Python 2.7, used to enhance deep conceptual learning in college courses.}%
\cvitem{}{\faicon{globe}\, \url{go.osu.edu/consider}}

\cventry{\faCode}{GeoGame}{}{}{}{A web based role playing game, developed using
  .NET, where players pose as farmers in various parts of the world, used to
  teach world regional geography at college level.}%
\cvitem{}{\faicon{globe}\, \url{geogame.osu.edu}}

\cventry{\faCode}{GDFA}{}{}{}{A generic data-flow analyzer plugin for GCC
  compilers, developed in C.}%
\cvitem{}{\faicon{globe}\, \url{cse.iitb.ac.in/grc}}

\end{document}

%%% Local Variables:
%%% mode: latex
%%% TeX-master: t
%%% TeX-engine: xetex
%%% End:
