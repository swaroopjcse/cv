\documentclass[11pt,a4paper]{moderncv}

%% ModernCV themes
% \moderncvstyle{banking}
% \moderncvstyle{casual}
\moderncvstyle{classic}
\moderncvcolor{red}
\renewcommand{\familydefault}{\sfdefault}
% \nopagenumbers{}

%%------------------------------------------------------------------------------
\usepackage[sorting=ydnt, backend=biber, style=numeric, giveninits=true,
maxbibnames=99]{biblatex}

\defbibenvironment{bibliography}
  {\list
     {\printtext[labelnumberwidth]{% label format from numeric.bbx
        \printfield{prefixnumber}%
        \printfield{labelnumber}}}
     {\setlength{\topsep}{0pt}% layout parameters from moderncvstyleclassic.sty
      \setlength{\labelwidth}{\hintscolumnwidth}%
      \setlength{\labelsep}{\separatorcolumnwidth}%
      \leftmargin\labelwidth%
      \advance\leftmargin\labelsep}%
      \sloppy\clubpenalty4000\widowpenalty4000}
  {\endlist}
  {\item}

\addbibresource{/Users/swaroop/Documents/Professional/mypubs/mypub.bib}
% \defbibnote{conf}{{\Large Conference Proceedings}}
% \defbibnote{other}{{\Large Other publications}}
%%------------------------------------------------------------------------------

\usepackage{academicons}
\usepackage{fontawesome}
%%------------------------------------------------------------------------------
\renewcommand*{\mobilephonesymbol}    {{\faPhone}~}
\renewcommand*{\emailsymbol}          {{\faicon{envelope-o}~}}
\renewcommand*{\homepagesymbol}       {{\faicon{globe}~}}
\renewcommand*{\linkedinsocialsymbol} {{\faicon{linkedin}~}}
\renewcommand*{\addresssymbol}        {{\faicon{university}~}}
\renewcommand*{\twittersocialsymbol}  {{\faicon{twitter}~}}
\renewcommand*{\githubsocialsymbol}   {{\faicon{github}~}}
%%------------------------------------------------------------------------------
\definecolor{color1}{RGB}{187,0,0}% OSU Scarlet
\definecolor{color2}{RGB}{102,102,102}% OSU Gray
%%------------------------------------------------------------------------------

%% Character encoding
\usepackage[utf8]{inputenc}

%% Adjust the page margins
\usepackage[scale=0.75]{geometry}
\usepackage{fontspec}

%% Personal data
\firstname{Swaroop}
\familyname{Joshi}
% \title{Resumé title (optional)}
% \address{395 Dreese Labs, 2015 Neil Ave}{Columbus OH 43210}
\address{405 Caldwell Labs\\2024 Neil Ave\\}{Columbus OH 43210}
% \phone{+2~(345)~678~901}
% \fax{+3~(456)~789~012}
\email{joshi.127@osu.edu}
\homepage{go.osu.edu/swaroop}
% \mobile{+1~(917)~833~8905}
% \extrainfo{Updated: \VCDateRAW}
% \photo[64pt][0.4pt]{picture}
% \quote{Researcher, Developer, Educator}
\social[linkedin]{swaroop84}
% \social[github]{swaroopjcse}


%%------------------------------------------------------------------------------

\usepackage{fancyhdr}
\usepackage{lastpage}
\fancyfoot[L]{\VCAuthor}
\fancyfoot[R]{\footnotesize Page \thepage\ of \pageref{LastPage}}
\fancypagestyle{firststyle}
{
   \fancyhf{}
   \fancyfoot[L]{\footnotesize Updated: \VCDateRAW~(\VCRevision)}
   \fancyfoot[R]{\footnotesize Page \thepage\ of \pageref{LastPage}}
}
%%------------------------------------------------------------------------------

%% My macros
\newcommand{\con}{CONSIDER}
\newcommand{\confull}{CONflicting Student Ideas Discussed, Evaluated and Resolved}
\newcommand{\osu}{The Ohio State University}
\newcommand{\iitb}{Indian Institute of Technology Bombay}
\newcommand{\iitbs}{IIT Bombay}
\newcommand{\grc}{GCC Resource Center}
\newcommand{\nitk}{National Institute of Technology Karnataka}
\newcommand{\nitks}{NITK}
\newcommand{\cse}{Computer Science \& Engineering}
\newcommand{\gra}{Graduate Research Associate}
\newcommand{\gta}{Graduate Teaching Associate}
\newcommand{\geogame}{\url{geogame.osu.edu}}
\newcommand{\thesis}{\con: A Novel Approach to Conflict-Driven Collaborative-Learning}
\newcommand{\gdfa}{Extending the Generic Data-Flow Analyzer (gdfa) in GCC}
\newcommand{\ucat}{University Center for Advancement in Teaching}
%%------------------------------------------------------------------------------

\immediate\write18{sh ./vc}
\input{vc}
%%------------------------------------------------------------------------------
%% Content
%%------------------------------------------------------------------------------
\begin{document}
\makecvtitle
\thispagestyle{firststyle}
\section{Education}
\cventry{2017}{Doctor of Philosophy}{\osu}{Columbus OH}{}{\cse}  % arguments 3 to 6 can be left empty
\cvitem{thesis}{\emph{\thesis}}
% \cvitemwithcomment{Language 1}{Skill level}{Comment}
% \cvdoubleitem{category X}{XXX, YYY, ZZZ}{category Y}{XXX, YYY, ZZZ}
% \cvlistitem{Item 1}
% \cvlistdoubleitem{Item 2}{Item 3}
%% ...
\cventry{2016}{Master of Science}{\osu}{Columbus OH}{}{\cse}
\cventry{2010}{Master of Technology}{\iitb}{Mumbai India}{}{\cse}
\cvitem{project}{\emph{\gdfa}}
\cventry{2005}{Bachelor of Engineering}{\nitk}{India}{}{Computer Engineering}

\section{Professional Experience}
% \label{sec:Experience}
\cventry{}{Senior Lecturer}{Dept.\ of Computer Sci.\ Eng.,\osu}{2017---}{}{}
\cventry{}{\gta}{\osu}{2013--14, 2016}{}{}
\cventry{}{\gra}{\osu}{2011--12, 2014--16}{}{}
% \cvitemwithcomment{}{\gta, \osu}{2012--14, 2016---}
% \cvitemwithcomment{}{\gra, \osu}{2011--12, 2014--16}
% \cventry{2011--12}{\gra}{\osu}{}{}{}
\cventry{}{Senior Project Engineer}{\grc, \iitbs}{2010--11}{}{}
% \cvitemwithcomment{}{Senior Project Engineer, \grc, \iitbs} {2010--11}
\cventry{}{Teaching Assistant}{\iitbs}{2008--10}{}{}
% \cvitemwithcomment{}{Teaching Assistant, \iitbs}{2008--10}
\cventry{}{Software Engineer}{SoftJin Technologies}{Bangalore India}{2005--06}{}
% \cvitemwithcomment{}{Software Engineer, SoftJin Technologies, Bangalore India}{2005--06}

%%------------------------------------------------------------------------------
\section{Research/Teaching Interests}
\cvitem{}{Programming Languages, Compiler Construction, Mobile and Web App Development, Software Engineering, Introduction to Programming, Education Technology, Computer supported collaborative learning (CSCL), Game based learning, Computer Science Education, Engineering Education}
\clearpage

%%------------------------------------------------------------------------------
\section{Teaching Experience}
\cventry{Jan. 2017 -- }{Instructor}{Software I: Components}{OSU}{}{}
\cvitem{}{Over the span of three semesters (Spring, Summer, and Fall 2017), I
  have taught total of 7 sections of this pre-major course, with about 40
  students each. This course focuses on Intellectual foundations of software
  engineering; design-by-contract principles; mathematical modeling of software
  functionality; component-based software from client perspective. }

\cventry{Summer 2017}{Instructor}{Introduction to Computer Programming In
  Java}{OSU}{}{}
\cvitem{}{This course introduces students to computer programming and to problem solving techniques using computer programs; taken by CSE majors as well as non-majors.}

\cventry{Fall 2016}{Instructor}{Data Structures Using Java}{OSU}{}{}
\cvitem{}{Teaching a sophomore level that is designed to introduce the general
  concepts of computer programming and programming languages by providing
  practical experience programming in the Java programming language. The topics
  focused are: Subroutines and modular programming, searching, basic data
  structures, introduction to sequential files.}

\cventry{Spring \& Summer 2014}{Instructor}{Mobile App Development}{OSU}{}{}
\cvitem{}{Taught a senior/graduate level project-based course on mobile app development for two semesters}
\cvitem{}{Revised course content to suit newer versions of the Android operating system}
\cvitem{}{Advised student teams in developing 10--12 projects each semester}
\cvitem{}{Designed and graded 40+ midterm and final exams per semester}
\cvitem{}{Designed and developed multiple apps to explain the concepts in the course}

\cventry{Fall 2014}{Grader}{Software-I}{OSU}{}{}
\cvitem{}{Graded 40+ assignments in this introductory course on software and computer programming}

\cventry{Summer 2013}{Instructor}{C++ Programming}{OSU}{}{}
\cvitem{}{Taught a senior level C++ course, focusing on the object oriented nature of the language}
\cvitem{}{Designed lecture and presentation material for the course}
\cvitem{}{Designed and graded 20+ assignments}

\cventry{Spring 2013}{Instructor}{Introduction to Programming}{OSU}{}{}
\cvitem{}{Taught a freshman/sophomore level programming course using C++; most students were not computer science majors}
\cvitem{}{Conducted the corresponding lab for the course (2 hours per week)}
\cvitem{}{Graded 40+ lab assignments, homeworks and exams}
\cvitem{}{Communicated with other TAs and the course-coordinator in this 10-section course}

\cventry{Summer 2013}{Grader}{Operating Systems}{OSU}{}{}
\cvitem{}{Graded 40+ assignments and homeworks in this junior/senior level course}

\cventry{2009--2010}{Grader}{Program Analysis}{\iitbs}{}{}
\cvitem{}{Graded 40+ assignments and exams in this graduate level computer science course for two semesters}

\cventry{2008}{Grader}{Introduction to programming}{\iitbs}{}{}
\cvitem{}{Conducted programming labs for this introductory programming course for non-majors and majors (4 hours per week)}
\cvitem{}{Graded 40+ assignments and exams}

\clearpage

%%------------------------------------------------------------------------------
\section{Research Experience}
% \cventry{}{Ph.D.\ Thesis}{\osu}{}{}{}
\subsection{Ph.D.\ Thesis}
\cvitem{title}{\emph{\thesis}}
\cvitem{advisor}{Prof.\ Neelam Soundarajan, \cse, OSU}
% \cvitem{description}{\con\ is a novel, powerful approach for effective
% collaborative learning in college courses, particularly in computer science
% and other STEM disciplines. The design of the approach is informed by research
% in education psychology, and its key features are designed to overcome some
% known issues in other collaborative learning approaches. We have also
% developed a responsive web app based on this approach, using Google App Engine
% and Python 2.7, which is available as a free and open source software. We are
% using it in several undergrad/grad level CS courses for its evaluation and
% development.}

\cvitem{description}{Piaget's classic work on cognitive development showed that
  engaging learners in critical discussions with peers about ideas that are
  different than theirs leads to deep conceptual understanding. Implementing
  such an approach in college level STEM (Science, Technology, Engineering,
  Math) courses has some specific challenges: (a) Short meeting times and large
  class sizes; (b) Competitive nature of the courses and single answer questions
  on assignments and exams; and (c) Overall lack of collaborative learning
  culture where students are unsure of how to seek help and many faculty members
  tend to think that engaging in collaborative activities may affect content
  coverage; etc. Based on Piaget's theory, I have developed a highly innovative
  collaborative learning approach that exploits specific affordances of web
  technologies to address these challenges. This approach, named CONSIDER (short
  for CONflicting Ideas Discussed, Evaluated, and Resolved) allows creation of
  small groups of students with different ideas about the topic in question,
  engages them in a highly-structured rounds-based discussion so that the group
  progresses at an equitable pace, and makes their submissions anonymous to
  others so that students can receive the comments without any preconceived
  notions they may have about the poster. While a number of researchers have
  explored approaches to collaborative learning, a key difference with this work
  is that my focus is helping individual students develop their own
  understanding, whereas the focus of much of this other work is on developing
  students' team skills, effective communication abilities, and the like.}
\cvitem{}{A platform independent responsive web application was developed to
  implement this approach and to compare it with the commonly used discussion
  forum approach, where students respond to other posts in a threaded
  discussion. This app was used in three offerings of two junior/senior level
  undergraduate Computer Science \& Engineering courses at The Ohio State
  University, where one discussion was conducted using the forum-based approach
  and the other using my approach in each course. Students performance on the
  pre- and post-discussion question and participation of individual students was
  measured. In two of the studies, significant difference was found in the
  discussion that used the CONSIDER approach, compared to the forum-based
  approach on both these measures: improvement in learning (Study 1:
  $N=37,r=−.63,p<.05$; Study 2: $N=26,r=−.38,p<.05$) and participation (Study 1:
  $r=−.57,p<.05$; Study 2: $r=−.76,p<.05$). In the third study, there was no
  significant effect on the post-activity scores ($p=.37$), and the
  participation in the two types of discussions did not differ significantly,
  either ($r=−.20,p=.09$). In an anonymous, optional post-activity survey, a
  majority of the students self reported that they found the rounds-based
  structure (55\%) and anonymity (80\%) to be helpful in their participation and
  learning.}

  \cvitem{\faicon{globe}}{\url{http://go.osu.edu/consider}}

\subsection{Other Research Projects}
\cventry{2014--2016}{\gra}{\osu}{}{}{}

\cvitem{}{Developed functionality for GeoGame, a web based role-playing game
  developed using MVC .NET, which lets the students of geography experience the
  life of an ordinary farmer in developing countries.}

\cvitem{}{Coordinated the teams of developers, adding functionality to the game,
  designing experiments to evaluate effects of the game on learning, and
  resolving software issues.}

\cvitem{}{Led multiple capstone teams during the course of these projects.}
% \cvitem{\faicon{globe}}{\url{http://geogame.osu.edu}}

\cventry{2011--2012}{\gra}{\osu}{}{}{}
\cvitem{}{Worked on testing and evaluation of a software PolyOpt/Fortran, which is a locally developed loop optimization framework}

\cventry{2010--2011}{Senior Project Engineer}{\grc, \iitbs}{}{}{}
\cvitem{}{Adapted the Generic Data-Flow Analyzer in GCC 4.6, making it available as a plugin}

% \clearpage
% \cvitem{\aiicon{orcid}}{\url{http://orcid.org/0000-0003-4536-2446}}
%%------------------------------------------------------------------------------
% \renewcommand{\refname}{Some Relevant Publications}
\nocite{*}
\printbibliography[title=Book Chapters, type=incollection]
\printbibliography[title=Conference Proceedings, type=inproceedings]
\printbibliography[title=Other Publications, nottype=inproceedings, nottype=incollection]
%%------------------------------------------------------------------------------
% \section{Skills}
% \cvitem{Languages}{Python, HTML, Javascript, C, C++, C\#, Java, R}
%
% \cvitem{Softwares}{Pycharm IDE, Eclipse, Microsoft Visual Studio, git, gcc}
%--------------------------

\section{Talks/Presentations}
\cventry{Aug.\ 2016}{New TA Orientation Facilitator}{\ucat}{\osu}{}{Served as a facilitator for the session on `Teaching as a Grader' and as a panelist on the session for `International Teaching Associates'}

\section{Awards and Honors}
\cventry{}{Best Student Paper Award}{ASEE North Central Section}{2016}{}{}
\cventry{}{Student Member}{American Society for Engineering Education}{}{}{}
\cventry{}{Student Member}{Association for Computing Machinery}{}{}{}

\clearpage
%%------------------------------------------------------------------------------
\section{Service Activities}
\subsection{Field/Profession}
\cventry{2016}{Reviewer}{\href{http://news.uitm.edu.my/index.php/n/research/247-8th-international-conference-on-university-learning-and-teaching-incult-2016}{InCULT'2016}}{}{}{I was a reviewer for the 8th International Conference on University Learning and Teaching (InCULT 2016), where I reviewed a full length research article for its scholarly contribution and provided my feedback.}
\cventry{2016}{Reviewer}{Frontiers in Education}{}{}{I was a reviewer for IEEE's Frontiers in Education 2016 conference, where I reviewed two full length research articles for their scholarly contribution and provided my feedback.}
\cventry{2016}{Volunteer}{Buck-i-code}{ACM-W, OSU}{}{Association for Computing and Machinery - Women's OSU chapter organized a day-long workshop Buck-i-Code in Spring 2016, targeted at school-going female students to get them interested in computer science at a young age. I was one of the volunteers who helped them with their lab sessions, as well as registrations.}
\cventry{2009, 2010}{Technical assistant}{\grc}{\iitbs}{}{The \grc\ organizes a workshop ``Essential Abstractions in GCC'' every year in July. I was part of the technical team, in charge of conducting the lab sessions for two workshops.}


\subsection{University}
\cventry{Mar.\ 2015}{Research Poster Judge}{Denman Undergraduate Research Forum}{OSU}{}{}
\cventry{2015---}{Graduate Representative}{Mobile Governance}{OSU}{}{Mobile Governance initiative at OSU is a collaborative effort between OCIO, TCO, and other stakeholders at OSU to ensure the large number of mobile and web applications developed by various departments at OSU (e.g.\ medical center) meet the quality and branding standards of the university. I am part of this initiative as a CSE graduate student representative since April 2015. My role includes participating in regular meetings and communicating with the capstone course coordinators in the CSE department to ensure any product coming out of their course does not violate the university guidelines.}


%%------------------------------------------------------------------------------

\section{Softwares Developed}

\cventry{\faicon{code}}{\con}{}{}{}{A web app, developed using Google App Engine and Python 2.7, used to enhance deep conceptual learning in college courses.}
\cvitem{}{\faicon{globe}\, \url{go.osu.edu/consider}}

\cventry{\faCode}{GeoGame}{}{}{}{A web based role playing game, developed using .NET, where players pose as farmers in various parts of the world, used to teach world regional geography at college level.}
\cvitem{}{\faicon{globe}\, \url{geogame.osu.edu}}

\cventry{\faCode}{GDFA}{}{}{}{A generic data-flow analyzer plugin for GCC compilers, developed in C.}
\cvitem{}{\faicon{globe}\, \url{cse.iitb.ac.in/grc}}

\end{document}

%%% Local Variables:
%%% mode: latex
%%% TeX-master: t
%%% TeX-engine: xetex
%%% End:
