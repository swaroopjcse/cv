\documentclass[11pt,a4paper]{moderncv}

%% ModernCV themes
% \moderncvstyle{fancy}
% \moderncvstyle{casual}
\moderncvstyle{classic}
\moderncvcolor{blue}
\renewcommand{\familydefault}{\sfdefault}
% \nopagenumbers{}

%% Character encoding
\usepackage[utf8]{inputenc}

%% Adjust the page margins
\usepackage[scale=0.75]{geometry}
\usepackage{fontspec}
\usepackage{fontawesome}

%% Personal data
\firstname{Swaroop}
\familyname{Joshi}
% \title{Resumé title (optional)}
% \address{395 Dreese Labs, 2015 Neil Ave}{Columbus OH 43210}
\address{661 Tuscarawas Ct}{Columbus OH 43210}
% \phone{+2~(345)~678~901}
% \fax{+3~(456)~789~012}
\email{joshi.127@osu.edu}
\homepage{go.osu.edu/swaroop}
\mobile{+1~(917)~833~8905}
\extrainfo{Updated: \VCDateRAW}
% \photo[64pt][0.4pt]{picture}
\quote{Researcher, Developer, Educator}
\social[linkedin]{swaroop84}
% \social[github]{swaroopjcse}


%%------------------------------------------------------------------------------
%% My macros
\newcommand{\con}{CONSIDER}
\newcommand{\confull}{CONflicting Student Ideas Discussed, Evaluated and Resolved}
\newcommand{\osu}{The Ohio State University}
\newcommand{\iitb}{Indian Institute of Technology Bombay}
\newcommand{\iitbs}{IIT Bombay}
\newcommand{\nitk}{National Institute of Technology Karnataka}
\newcommand{\nitks}{NITK}
\newcommand{\cse}{Computer Science \& Engineering}
\newcommand{\gra}{Graduate Research Associate}
\newcommand{\gta}{Graduate Teaching Associate}
\newcommand{\geogame}{\url{geogame.osu.edu}}
\newcommand{\thesis}{\con: A Novel Approach to Conflict-Driven Collaborative-Learning}
%%------------------------------------------------------------------------------

\immediate\write18{sh ./vc}
\input{vc}
%%------------------------------------------------------------------------------
%% Content
%%------------------------------------------------------------------------------
\begin{document}
\makecvtitle

\section{Education}
\cventry{2011---}{Ph.D.}{\osu}{Columbus OH}{}{\cse}  % arguments 3 to 6 can be left empty
% \cvitem{title}{ \emph{Title} }
% \cvitemwithcomment{Language 1}{Skill level}{Comment}
% \cvdoubleitem{category X}{XXX, YYY, ZZZ}{category Y}{XXX, YYY, ZZZ}
% \cvlistitem{Item 1}
% \cvlistdoubleitem{Item 2}{Item 3}
%% ...
\cventry{2016}{Master of Science}{\osu}{Columbus OH}{}{\cse}
\cventry{2010}{Master of Technology}{\iitb}{Mumbai India}{}{\cse}
\cventry{2005}{Bachelor of Engineering}{\nitk}{India}{}{Comupter Engineering}

\section{Experience}
% \label{sec:Experience}
\cventry{2014---}{\gra}{\osu}{}{}{Software developer and researcher on the GeoGame project (\geogame), which is a web based role-playing game developed using MVC .NET that lets the students of geography experience the life of an ordinary farmer in developing countries.}

\cventry{2012--14}{\gta}{\osu}{}{}{Instructed three CS courses: Mobile App Development, Introduction to Computer Programming, and C++ Programming
\newline{}Grader for two other courses: Operating Systems and Software-I}

\cventry{2011--12}{\gra}{\osu}{}{}{Worked on testing and analyzing PolyOpt/Fortran, a loop optimization framework.}

\cventry{2010--11}{Senior Project Engineer}{GCC Resource Center, \iitbs}{}{}{Adapting the Generic Data Flow Analyzer for GCC 4.6, making it available as a plugin}

\cventry{2005--06}{Software Engineer}{SoftJin Technologies}{Bangalore India}{}{Worked on translating the OASIS format designs to the OpenAccess format}

%%------------------------------------------------------------------------------
\section{Research/Teaching Interests}
\cvitem{}{Programming Languages, Compiler Construction, Mobile and Web App Development, Software Engineering, Introduction to Programming, Education Technology, Computer supported collaborative learning (CSCL), Game based learning, Computer Science Education, Engineering Education}
\clearpage

%%------------------------------------------------------------------------------
\section{Ph.D.\ Thesis}
\cvitem{title}{\emph{\thesis}}
\cvitem{advisor}{Prof.\ Neelam Soundarajan, \cse, OSU}
\cvitem{description}{\con\ is a novel, powerful approach for effective collaborative learning in college courses, particularly in computer science and other STEM disciplines. The design of the approach is informed by research in education psychology, and its key features are designed to overcome some known issues in other collaborative learning approaches. We have also developed a responsive web app based on this approach, using Google App Engine and Python 2.7, which is available as a free and open source software. We are using it in several undergrad/grad level CS courses for its evaluation and development.}
\cvitem{website}{\url{http://go.osu.edu/consider}}
%%------------------------------------------------------------------------------
\renewcommand{\refname}{Some Relevant Publications}
\nocite{*}
\bibliographystyle{abbrv}
\bibliography{mypub}

%%------------------------------------------------------------------------------
\section{Skills}
\cvitem{Languages}{Python, HTML, Javascript, C, C++, Java, R}

\cvitem{Softwares}{Pycharm IDE, Eclipse, Microsoft Visual Studio, Git, gcc}
%--------------------------

\section{Honors}
\cvitem{}{American Society for Engineering Education---Student Member}

%%------------------------------------------------------------------------------
\section{Awards}
\cvitem{}{Best Student Paper Award---ASEE North Central Section (2016)}

%%------------------------------------------------------------------------------

\section{Softwares Developed}

\cvitem{CONSIDER}{A web app, developed using Google App Engine and Python 2.7, used to enhance deep conceptual learning in college courses. \faicon{globe}\, \url{go.osu.edu/consider}}
\cvitem{GeoGame}{A web based role playing game, developed using .NET, where players pose as farmers in various parts of the world, used to teach world regional geography at college level. \faicon{globe}\, \url{geogame.osu.edu}}
\cvitem{GDFA}{A generic data-flow analyzer plugin for GCC compilers (v~4.5--6), developed in C. \mbox{\faicon{globe}\, \url{cse.iitb.ac.in/grc}}}

%%------------------------------------------------------------------------------
%-----       letter       ---------------------------------------------------------
% recipient data
\recipient{Company Recruitment team}{Company, Inc.\\123 somestreet\\some city}
\date{\today}
\opening{Dear Sir or Madam,}
\closing{Yours faithfully,}
\enclosure[Attached]{Résumé}          % use an optional argument to use a string other than "Enclosure", or redefine \enclname
\makelettertitle

Lorem ipsum dolor sit amet, consectetur adipiscing elit. Duis ullamcorper neque sit amet lectus facilisis sed luctus nisl iaculis. Vivamus at neque arcu, sed tempor quam. Curabitur pharetra tincidunt tincidunt. Morbi volutpat feugiat mauris, quis tempor neque vehicula volutpat. Duis tristique justo vel massa fermentum accumsan. Mauris ante elit, feugiat vestibulum tempor eget, eleifend ac ipsum. Donec scelerisque lobortis ipsum eu vestibulum. Pellentesque vel massa at felis accumsan rhoncus.

Suspendisse commodo, massa eu congue tincidunt, elit mauris pellentesque orci, cursus tempor odio nisl euismod augue. Aliquam adipiscing nibh ut odio sodales et pulvinar tortor laoreet. Mauris a accumsan ligula. Class aptent taciti sociosqu ad litora torquent per conubia nostra, per inceptos himenaeos. Suspendisse vulputate sem vehicula ipsum varius nec tempus dui dapibus. Phasellus et est urna, ut auctor erat. Sed tincidunt odio id odio aliquam mattis. Donec sapien nulla, feugiat eget adipiscing sit amet, lacinia ut dolor. Phasellus tincidunt, leo a fringilla consectetur, felis diam aliquam urna, vitae aliquet lectus orci nec velit. Vivamus dapibus varius blandit.

Duis sit amet magna ante, at sodales diam. Aenean consectetur porta risus et sagittis. Ut interdum, enim varius pellentesque tincidunt, magna libero sodales tortor, ut fermentum nunc metus a ante. Vivamus odio leo, tincidunt eu luctus ut, sollicitudin sit amet metus. Nunc sed orci lectus. Ut sodales magna sed velit volutpat sit amet pulvinar diam venenatis.

Albert Einstein discovered that $e=mc^2$ in 1905.

\[ e=\lim_{n \to \infty} \left(1+\frac{1}{n}\right)^n \]

\makeletterclosing
\end{document}
